% cv_style.tex
% Comandi e stili per il CV

\usepackage{geometry}
\geometry{a4paper, left=1.5cm, right=1.5cm, top=1.5cm, bottom=1.5cm}

\usepackage{fontspec}
\setmainfont{Lato} % Un font moderno e leggibile

\usepackage{xcolor}
\usepackage{fontawesome5}
\usepackage{simpleicons} % Aggiunto per le icone tech
\usepackage{hyperref}
\usepackage{tabularx}
\usepackage{enumitem}
\usepackage{graphicx}
\usepackage{titlesec}

% Importazione dei colori
% ===============================================================================
% CV_COLORS.TEX - Sistema di colori modulare e scalabile per CV LaTeX
% ===============================================================================
% Versione: 2.0
% Descrizione: Palette colori professionale con temi intercambiabili
% ===============================================================================

% ----- PALETTE COLORI PRINCIPALE -----
% Palette Modern Professional (default)
\definecolor{primaryColor}{HTML}{2c3e50}      % Grigio-blu scuro - Titoli principali
\definecolor{secondaryColor}{HTML}{34495e}    % Grigio-blu medio - Testi e dettagli  
\definecolor{accentColor}{HTML}{16a085}       % Verde acqua - Link e accenti
\definecolor{lightGray}{HTML}{ecf0f1}         % Grigio chiaro - Sfondi e separatori
\definecolor{darkGray}{HTML}{555555}          % Grigio scuro - Testo secondario
\definecolor{white}{HTML}{FFFFFF}             % Bianco puro
\definecolor{black}{HTML}{000000}             % Nero puro

% ----- PALETTE ALTERNATIVE (COMMENTATE) -----
% Uncomment per utilizzare palette alternative

% Palette Corporate Blue
% \definecolor{primaryColor}{HTML}{1f4e79}    % Blu corporate scuro
% \definecolor{secondaryColor}{HTML}{2c5f8a}  % Blu corporate medio
% \definecolor{accentColor}{HTML}{3498db}     % Blu accento brillante
% \definecolor{lightGray}{HTML}{f8f9fa}       % Grigio molto chiaro
% \definecolor{darkGray}{HTML}{495057}        % Grigio corporate

% Palette Creative Purple
% \definecolor{primaryColor}{HTML}{6c5ce7}    % Viola creativo
% \definecolor{secondaryColor}{HTML}{5f3dc4}  % Viola scuro
% \definecolor{accentColor}{HTML}{fd79a8}     % Rosa accento
% \definecolor{lightGray}{HTML}{f1f2f6}       % Grigio neutro
% \definecolor{darkGray}{HTML}{57606f}        % Grigio freddo

% ----- SISTEMA DI CONTRASTO SEMANTICO -----
% Colori semantici derivati per coerenza
\colorlet{textPrimary}{primaryColor}
\colorlet{textSecondary}{secondaryColor}
\colorlet{textMuted}{darkGray}
\colorlet{linkColor}{accentColor}
\colorlet{borderColor}{lightGray}
\colorlet{backgroundLight}{lightGray}

% ----- SISTEMA DI LUMINOSITÀ -----
% Varianti di luminosità per maggiore flessibilità
\colorlet{primaryLight}{primaryColor!70}
\colorlet{primaryDark}{primaryColor!130}
\colorlet{secondaryLight}{secondaryColor!70}
\colorlet{secondaryDark}{secondaryColor!130}
\colorlet{accentLight}{accentColor!70}
\colorlet{accentDark}{accentColor!130}

% ----- UTILITÀ COLORI -----
% Colori di stato per feedback visivo
\definecolor{successColor}{HTML}{27ae60}      % Verde successo
\definecolor{warningColor}{HTML}{f39c12}      % Arancione warning
\definecolor{errorColor}{HTML}{e74c3c}        % Rosso errore
\definecolor{infoColor}{HTML}{3498db}         % Blu informazione

% ----- FINE CV_COLORS.TEX -----
% tech_brand_colors.tex
% Colori ufficiali dei brand per le icone delle tecnologie

% Front-end
\definecolor{HTML5Color}{HTML}{E34F26}
\definecolor{CSS3Color}{HTML}{1572B6}
\definecolor{JavaScriptColor}{HTML}{F7DF1E}
\definecolor{VueColor}{HTML}{4FC08D}
\definecolor{AngularColor}{HTML}{DD0031}
\definecolor{ReactColor}{HTML}{61DAFB}
\definecolor{TypeScriptColor}{HTML}{3178C6}
\definecolor{BootstrapColor}{HTML}{7952B3}
\definecolor{TailwindColor}{HTML}{38B2AC}
\definecolor{ViteColor}{HTML}{646CFF}
\definecolor{SassColor}{HTML}{CC6699}

% Back-end
\definecolor{PHPColor}{HTML}{777BB4}
\definecolor{LaravelColor}{HTML}{FF2D20}
\definecolor{NodeColor}{HTML}{8CC84B}
\definecolor{ExpressColor}{HTML}{000000}
\definecolor{PythonColor}{HTML}{3776AB}

% Database
\definecolor{MySQLColor}{HTML}{4479A1}

% Version Control
\definecolor{GitColor}{HTML}{F05032}
\definecolor{GitHubColor}{HTML}{181717}
\definecolor{DockerColor}{HTML}{2496ED}

% Altri
\definecolor{LinuxColor}{HTML}{FCC624} % Aggiunto per i colori dei brand

% Impostazioni globali
\pagestyle{empty}
\setlist{leftmargin=*, label=, noitemsep}
\hypersetup{
    colorlinks=true,
    urlcolor=accentColor,
    linkcolor=accentColor,
}

% ----- COMANDI PER LA STRUTTURA -----

% Titolo di sezione
\titleformat{\section}
  {\Large\bfseries\color{primaryColor}}
  {}
  {0em}
  {}[\titlerule]
\titlespacing*{\section}{0pt}{1.5ex}{1ex}

% Header con nome e titolo
\newcommand{\cvheader}[2]{
    \begin{center}
        {\Huge\bfseries\color{primaryColor} #1}\\[2mm]
        {\Large\color{secondaryColor} #2}
    \end{center}
    \vspace{5mm}
}

% Contatto nella sidebar
\newcommand{\contactitem}[2]{%
    \parbox[t]{0.9\linewidth}{%
        \textcolor{primaryColor}{\faIcon{#1}}\hspace{3mm}%
        \href{#2}{\textcolor{secondaryColor}{\hyphenchar\font=\defaulthyphenchar#2}}%
    }%
}
\newcommand{\contactitemtext}[2]{%
    \parbox[t]{0.9\linewidth}{%
        \textcolor{primaryColor}{\faIcon{#1}}\hspace{3mm}%
        \textcolor{secondaryColor}{#2}%
    }%
}

% Sezione della sidebar
\newcommand{\sidebarsection}[1]{
    \vspace{5mm}
    {\large\bfseries\color{primaryColor} #1}
    \rule[0.5ex]{\linewidth}{0.5pt}
}

% Categoria di skill
\newcommand{\skillcategory}[1]{
    \vspace{2mm}
    {\bfseries\color{secondaryColor} #1}
}

% Comando per le skill con icona
\newcommand{\techskill}[3]{
    \textcolor{#2}{\simpleicon{#1}}\hspace{0.8mm}#3\hspace{1mm}
}

% Voce di esperienza lavorativa
\newcommand{\cventry}[4]{
    \vspace{4mm}
    \textbf{\large\color{secondaryColor} #1} \hfill \textit{\color{darkGray} #2} \\
    \textbf{\color{accentColor} #3} \\
    \begin{itemize}
        #4
    \end{itemize}
}

% Voce di progetto
\newcommand{\cvproject}[3]{
    \vspace{4mm}
    \textbf{\large\color{secondaryColor} #1} \\
    \textit{\small\color{darkGray} #2} \\
    \begin{itemize}
        \item #3
    \end{itemize}
}

% Voce per formazione
\newcommand{\educationentry}[3]{
    \vspace{2mm}
    \textbf{\color{secondaryColor} #1} \\
    \textit{\color{darkGray} #2} \\
    \textit{\small #3}
}
